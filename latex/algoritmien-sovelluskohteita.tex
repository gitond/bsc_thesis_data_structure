\chapter{Algoritmien sovelluskohteita} \label{algoritmienSovelluskohteita}

\section{Videopelit}\label{videopelit}
Monet nykyaikaiset polunetsintäalgoritmeihin liittyvät tutkimukset tehdään 
videopeliteollisuuden 
käyttöön\cite{MathewAndMalathy}\cite{ACMHindawi}\cite{mazeGameTrilogi}. 
Tämä johtuu siitä, että videopeleissä törmätään usein tilanteisiin, jossa 
NPC-hahmojen pitää liikkua videopelikartalla niin, että pelaaja ei suoraan 
itse ohjaa niitä. Tällöin pelin on itse löydettävä reitti kahden pisteen 
välille. Kun polunetsintäalgoritmeja toteutetaan videopeleissä, joudutaan 
kiinnittämään huomiota muutamiin erityisvaatimuksiin. Polunetsintäalgoritmien 
ajaminen ei saa olla liian resurssi-intensiivistä, eli sen pitää pyrkiä 
kuluttamaan tietokoneen muistia ja suoritinaikaa mahdollisimman vähän. Jotta 
resurssi-intensiivisyydeltä vältyttäisiin, voidaan löystää polun 
optimaalisuuteen liittyviä vaatimuksia, jolloin polun ei tarvitse olla enää 
lyhin mahdollinen\cite{MathewAndMalathy}.\par
	Esimerkiksi reaaliaikaisissa strategiapeleissä (real-time strategy, 
RTS) pelaaja komentaa usein eri yksiköitä eri puolelle pelialuetta 
mekitsemällä niille maalipisteitä. Näitä yksiköitä voi olla jopa satoja, 
joille pitää löytää polut suurella etsintä-alueella huomioiden mahdollisesti 
muut alueella liikkuvat yksiköt, sekä pelin säännöt siitä miten eri 
yksiköt käytännössä voivat liikkua\cite{pPacman}\cite{MathewAndMalathy}. 
Nämä monimutkaiset laskutoimitukset on suoritettava pelin logiikan 
vaatimalla nopeudella. Esimerkiksi videopeliyhtiö BioWare on asettanut 
säännön, jonka mukaan kaikkien pelialueella liikkuvien agenttien polunetsintä 
on kyettävä suorittamaan 1-3 ms aikana\cite{pPacman}.\par
	\textcite{pPacman} kehittivät Pathfinding-in-Pacman-projektin, jossa 
he sovelsivat polunetsintäalgoritmeja Pac-Manin kaltaiseen peliin. Pelissä 
pelihahmo liikkuu labyrintissä yrittäen kerätä pisteitä, samalla kun sitä 
jahtaavat vihollishahmoina toimivat haamut. Pelaaja voittaa kerättyään kaikki 
pisteet, ja häviää kun haamut koskevat häntä. Peli sisältää toteutukset BFS- 
ja A*-algoritmeista, sekä pelin tarkoituksiin muokatusta A*-algoritmistä, 
jota projektin raportti kutsuu Context Dependent Subgoaling A*-algoritmiksi
(kontekstiriippuvaisesti osatavoitteistava A*, CDSA*) ja ohjelmakoodi kutsuu 
subGoalAStar-algoritmiksi. CSDA* etsii polun haamulta pelaajahahmolle 
A*-algoritmilla, mikäli haamu on tarpeeksi lähellä pelaajahahmoa. Muussa 
tapauksessa CSDA* palauttaa ainoastaan suunnan, jolla haamu pääsee lähemmäs 
pelaajahahmoa. Tällä tavoin säästetään järjestelmän resursseja rajoittamalla 
tehtävien laskujen määrää. Pelissä käytetään polunetsintäalgoritmeja 
haamujen ja pelaajahahmojen välisen polun löytämiseksi ja A*- ja 
CSDA*-algoritmit käyttävät heuristiikkana Manhattan-etäisyyttä. \par
	\textcite{SturtevantDAO} vertaavat tutkimuksessaan erilaisia 
suurelle etsintäalueelle soveltuvia polunetsintätekniikoita Dragon Age: 
Origins-videopelissä (DAO) käytettyyn polunetsintään. Nämä 
polunetsintätekniikat ovat abstraktiahierarkien käyttö, parempien 
heurististen funktioiden käyttö, sekä sopimushierarkioiden (Contrarction 
Hierarchy, CH) käyttö. Abstraktihierarkia ja CH ovat hierarkisessa 
polunetsinnässä (luku \ref{hpa}) käytettäviä tapoja muodostaa ylätason 
graafi. Abstraktihierarkia muodostaa ylätason graafin niin, että 
varsinaisen etsintäalueen eri osa-alueet ovat ylätason graafin solmuja, kun 
taas CH muodostaa ylätason graafin laskemalla alkuperäisen graafin jokaiselle 
solmulle tärkeystason (importance level) ja poistamalla vähemmän tärkeät 
solmut graafista. \textcite{SturtevantDAO} osoittivat, että hierarkinen 
polunetsintä joko abstraktiohierarkiaa tai CH:ta käyttäen soveltui parhaiten 
polunetsintään DAO:n kaltaisissa videopeleissä, koska näillä menetelmillä 
käytettiin vähiten suoritinta ja muistia. DAO käyttää abstraktiohierarkiaa, 
mutta \textcite{SturtevantDAO} mainitsevat, että pelin polunetsintää olisi 
voinut tehostaa entisestään käyttämällä suurempia osa-alueita ylätason 
graafin rakentamiseen.

\section{Karttaohjelmat}\label{karttaohjelmat}
Polunetsintäalgoritmit ovat myös keskeisessä osassa karttaohjelmistoissa. 
Monet karttaohjelmat tarjoavat reitinhakupalveluita, joissa generoidaan kartan  
eri pisteiden välille reitti käyttäen polunetsintäalgoritmeja. Tämä on usein 
helppo toteuttaa, koska karttaohjelmat säilyttävät karttadataa usein 
polunetsintäalgoritmeille ideaalisessa muodossa. Esimerkiksi OpenDrive-
formaatti sisältää dataa risteyksistä ja teistä, josta kootaan 
reititysgraafi\cite{Lanelet2}. \par
	\textcite{OpenStreetMap} on yhteisön ylläpitämä avoimen lähdekoodin 
karttatietokanta, joka tarjoaa sivustollaan myös tietokannan karttadataa 
käyttävää karttaohjelmaa\cite{OSMAbout}. OpenStreetMap-karttaohjelma sisältää 
myös reitinhakumahdollisuuden. Reitinhakumahdollisuus sisältää 
reitinhakuvaihtoehdot kävelylle, pyöräilylle ja autoilulle käyttäen joko 
GraphHopperia, OSRM:ää tai Valhallaa\cite{OpenStreetMap}. \par
	Nämä ovat niin sanottuja reititysmoottoreita (routing engine), jotka 
ovat valmiita ohjelmistototeutuksia polunetsintäfunktioille\cite{graphhopper}. 
OpenStreetMapin käyttämä \textcite{graphhopper} on Javalla kirjoitettu 
reititysmoottori. Se käyttää sisäisessä toteutuksessaan Dijkstran algoritmia, 
A*-algoritmia, sekä hierarkista polunetsintää käyttäen CH-menetelmää ylätason 
graafin luomiseen ja Dijkstran algoritmiä polunetsintään. GraphHopper valitsee 
ongelmaan sopivan polunetsintämenetelmän sen asetusten perusteella.

\section{Robotiikka}\label{robotiikka}
Polunetsintää tutkitaan myös robotiikan näkökulmasta, koska käytännössä 
kaikki robotit, jotka liikkuvaat jossakin ympäristössä, tarvitsevat 
polunetsintäalgoritmeja tehdäkseen päätöksiä siitä, minne 
mennä\cite{ProcediaAStar}. Liikkuvien robottien polunetsintään liittyy 
monenlaisia haasteita. Yksi näistä on törmäyksien estäminen\cite{ACMHindawi}. 
Robotit voivat törmätessään aiheuttaa vakavaa vahinkoa itseensä, 
ympäristöönsä ja alueella liikkuviin ihmisiin\cite{ProcediaAStar}. Esimerkiksi 
itseajavien autojen tapauksessa törmäykset voidaan pyrkiä estämään 
huomioimalla muut tienkäyttäjät paikallisessa reitinsuunnittelussa. Tämä 
ongelma kuitenkin monimutkaistuu kun tarvitsee huomioida erilaisia 
tienkäyttäjiä. Muut tienkäyttäjät huomioiva paikallinen reitinhakualgoritmi on 
helpompi suunnitella moottoritielle, missä voidaan odottaa liikkuvan 
pelkästään autoja ja voidaan odottaa että kaikilla on tarpeeksi tilaa, kuin 
kaupunkiin, jossa on lisäksi jalankulkijoita ja pyöräilijöitä, sekä 
määrällisesti enemmän tienkäyttäjiä ja vähemmän tilaa\cite{Lanelet2}.\par
	Törmäysten estämisen lisäksi robotiikassa tärkeässä roolissa on myös 
graafigeneraatio. Graafigeneraatio on osa lähes jokaista polunetsintäongelmaa, 
koska etsintäalue pitää muuttaa graafiksi, jotta polunetsintäalgoritmia voisi 
käyttää. Graafigeneraatio kuitenkin korostuu robotiikan sovelluksissa, koska 
toisin kun esimerkiksi videopeleissä tai karttaohjelmissa, tietokone ei 
hallitse etsintäaluetta, eikä se välttämättä ole tietoinen esteiden 
sijainnista alueella, tai alueen muutoksista. Tätä varten robotiikassa 
käytetään usein niin sanottuja dynaamisia polunetsintäalgoritmeja, jotka 
kykenevät reagoimaan etsintäalueen muutoksiin. Esimerkkejä näistä ovat 
A*-algoritmiin perustuva D*-algoritmi\cite{applSciLawande} ja 
\textcite{DelaunayVoronoiAStar} kehittämä DFPA-algoritmi. \par
	\textcite{transportRobotAStar} esittelevät artikkelissaan 
kehittämäänsä halpaa kuljetusrobottia, joka käyttää polunetsintään 
A*-algoritmia. Robotille syötetään ensin graafidata, jonka perusteella se 
sitten etsii lyhimmän polun A*-algoritmilla ja seuraa sitä maaliin asti. 
Robotti ei käytä dynaamista polunetsintää, eikä se kykene reagoimaan 
etsintäalueella tapahtuviin muutoksiin, vaan sille syötetään alueella olevat 
esteet graafidatan mukana. Lisäksi robotti sijoittaa itsensä graafiin 
seuraamalla toiminta-alueella maahan kiinnitettyjä mustia viivoja. Kyseinen 
robotti ei siis mielestäni soveltuisi tosielämän kuljetusrobotiksi, koska se 
ei pysty reagoimaan ympäristönsä muutoksiin ja se vaatii esiasennettua 
infrastruktuuria toimiakseen. Tutkimus keskittyi kuitenkin sensorisijoitusten 
optimalisointiin ja käännösmekanismin ja -algoritmien tekemiseen realistisen 
kuljetusrobotin polunetsintää enemmän. \par
	\textcite{warehouse} vertailevat artikkelissaan 
polunetsintäalgoritmeja varastoissa työskenteleville teollisuusroboteille. 
Koska varastossa saattaa liikkua kerralla jopa kymmeniä tai satoja robotteja, 
kyseessä on moniagentillinen polunetsintäongelma. Lisäksi artikkeli olettaa, 
että robotit liikkuvat varastossa jatkuvasti lastausalueiden ja 
varastointialueiden välillä niin, että kun jokin robotti saa sille asetetun 
tehtävän valmiiksi, sille annetaan uusi satunnainen tehtävä. \par
	Robotin sijaintia ei siis kyetä ennustamaan nykyisen tehtävän 
suorittamisen jälkeen. Tämän vuoksi artikkelin polunetsintäalgoritmt ajetaan 
uudestaan sille asetetun aikaikkunan $\omega$ mukaan. Tällöin algoritmeille 
riittää, että löydetyt polut ovat törmäysvapaita seuraavan aikaikkunan alkuun 
asti, eli vähintään $\omega$ ajan. Tätä kutsutaan artikkelissa rullaavan 
horisontin törmäystenratkaisuksi (Rolling Horizon Collision Resolution, 
RHCR)\par
	Artikkelin vertaamat polunetsintäalgoritmit ovat ikkunoitu 
hierarkinen yhteistoimintakykyinen A* (Windowed Hierarchical Cooperative A*, 
WHCA*), tehostettu konfliktipohjainen haku (Enhanced Conflict Based Search, 
ECBS) ja prioriteettipohjainen haku (Priority Based Search, PBS). WCHA* on 
hierarkinen (luku \ref{hpa}) A*-algoritmi, joka on muokattu toimimaan 
moniagentillisten ongelmien kanssa ikkunoidusti. ECBS on tehostettu versio 
CBS-algoritmista, joka etsii polkua iteroituvasti niin, että aina kun 
törmäyskohta löydetään niin polunetsintä aloitetaan alusta niin, että 
törmäysmahdollisuus huomioidaan rajoitteena. PBS on yhdistelmä 
moniagentilliseen ympäristöön mukautettua A*-algortimia ja CBS-algoritmia. 
Artikkelin mittauksissa PBS oli marginaalisesti tehokkaampi kuin WHCA* ja 
ECBS oli niin hidas, että se ei pystynyt tutkimukseen asennettujen 
aikarajoitteiden mukaan ratkaisemaan kuin yhden kolmesta tutkitusta 
varastosta.
