\chapter{Algoritmien sovelluskohteita} \label{algoritmienSovelluskohteita}

\section{Videopelit}\label{videopelit}
Monet nykyaikaiset polunetsintäalgoritmeihin liittyvät tutkimukset tehdään 
videopeliteollisuuden 
käyttöön.\cite{MathewAndMalathy}\cite{ACMHindawi}\cite{mazeGameTrilogy} 
Tämä johtuu siitä, että videopeleissä törmätään usein tilanteisiin, jossa 
ei-pelattavien hahmojen (non-playable character, NPC) pitää liikkua 
videopelikartalla niin, että pelaaja ei suoraan itse ohjaa niitä. Tällöin 
pelin on itse löydettävä reitti kahden pisteen välille. Kun 
polunetsintäalgoritmeja toteutetaan videopeleissä, joudutaan kiinnittämään 
huomiota muutamiin erityisvaatimuksiin. Polunetsintäalgoritmien ajaminen 
ei saa olla liian resurssi-intensiivistä, eli sen pitää pyrkiä kuluttamaan 
tietokoneen muistia ja suoritinaikaa mahdollisimman vähän. Jotta 
resurssi-intensiivisyydeltä vältyttäisiin, voidaan löystää polun 
optimaalisuuteen liittyviä vaatimuksia, jolloin polun ei tarvitse olla enää 
lyhin mahdollinen.\cite{MathewAndMalathy} \par
	Esimerkiksi reaaliaikaisissa strategiapeleissä (real-time strategy, 
RTS) pelaaja komentaa usein eri yksiköitä eri puolelle pelialuetta 
mekitsemällä niille maalipisteitä. 

\section{Karttaohjelmat}\label{karttaohjelmat}
\section{Robotiikka}\label{robotiikka}
