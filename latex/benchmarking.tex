\chapter{Eräiden algoritmien tehokkuuden tarkastelu esimerkkiongelmassa}\label{benchmarking}

Tässä tutkielmassa on verrattu kahta polunetsintäalgoritmia, Dijkstran 
algoritmia ja BFS:ää, ajamalla niitä koneellisesti monta kertaa satunnaisesti 
generoiduissa graafeissa ja tutkimalla ajautumisaikoja. Graafit on toteutettu
Boost Graph Library C++ kirjaston avulla jonka kehitti \textcite{bgl} . 
Testiohjelma ajaa algoritmit kahdessa Boost Graph Libraryn eri 
graafitoteutuksessa. Tämän tarkoitus on demonstroida graafien toteutuksen 
vaikutusta algoritmeihin. Testiohjelma itsessään on kehitetty tätä 
projektia varten ja julkaistu kokonaisuudessaan avoimen lähdekoodin jakeluun.
\cite{gt2} \par
	Testiohjelma ratkaisee jokaisella testikerralla 2000 
polunetsintäongelmaa, jossa jokaista varten generoidaan graafi $n$ solmulla 
ja $1,25n$ kaarella. Ajokertoja oli yhteensä 200/kategoria, eli jokaista 
kategoriaa kohtaan ratkaistiin yhteensä $2000*200 = 40 000$ 
polunetsintäongelmaa. Käytetyt graafikoot olivat $n=64$, $n=128$ ja $n=512$. 
Toteutuksen testisilmukassa oli kutsu, sekä BFS:lle, että Dijkstran 
algoritmille, sekä listatyyppisessä-, että matriisityyppisessä graafissa, 
mutta yksittäisissä testeissä kommentoitiin kaikki muu kuin testattava. Täten 
kun testattiin esimerkiksi Dijkstran algoritmia matriisityyppisessä graafissa, 
niin kaikki listatyyppisiin graafeihin ja BFS:ään liittyvät toimenpiteet oli 
kommentoitu pois, jolloin ne eivät vaikuta testien tuloksiin.\cite{gt2} \par

\begin{table}
\centering{}\caption{Yhteenveto testeissä mitatuista ajoista (sekuntia)\label{tulokset}}
\begin{tabular}{l|c|c|}
Graafien tyyppi ja koko (solmuja) & BFS & Dijkstran algoritmi \tabularnewline
\hline 
				Lista 64 	& 
\begin{tabular}{@{}c@{}}	Keskiarvo: 2,805 \\ Keskihajonta: 0,3972	\end{tabular} & 
\begin{tabular}{@{}c@{}}	Keskiarvo: 2,975 \\ Keskihajonta: 0,1565	\end{tabular} \tabularnewline
\hline
				Lista 128 	& 
\begin{tabular}{@{}c@{}}	Keskiarvo: 5,345 \\ Keskihajonta: 0,4766	\end{tabular} & 
\begin{tabular}{@{}c@{}}	Keskiarvo: 6,63  \\ Keskihajonta: 0,4840	\end{tabular} \tabularnewline
\hline
				Lista 512 	& 
\begin{tabular}{@{}c@{}}	Keskiarvo: 25,67 \\ Keskihajonta: 0,4714	\end{tabular} & 
\begin{tabular}{@{}c@{}}	Keskiarvo: 50,4  \\ Keskihajonta: 0,5931	\end{tabular} \tabularnewline
\hline
				Matriisi 64 	& 
\begin{tabular}{@{}c@{}}	Keskiarvo: 3,185 \\ Keskihajonta: 0,3893	\end{tabular} & 
\begin{tabular}{@{}c@{}}	Keskiarvo: 2,945 \\ Keskihajonta: 0,2286	\end{tabular} \tabularnewline
\hline 
				Matriisi 128 	& 
\begin{tabular}{@{}c@{}}	Keskiarvo: 9,49  \\ Keskihajonta: 0,5012	\end{tabular} & 
\begin{tabular}{@{}c@{}}	Keskiarvo: 8,77  \\ Keskihajonta: 0,4219	\end{tabular} \tabularnewline
\hline 
				Matriisi 512 	& 
\begin{tabular}{@{}c@{}}	Keskiarvo: 124,73 \\ Keskihajonta: 0,6156	\end{tabular} & 
\begin{tabular}{@{}c@{}}	Keskiarvo: 114,315 \\ Keskihajonta: 1,214	\end{tabular} \tabularnewline
\end{tabular}
\end{table}

Taulukossa \ref{tulokset} esitetään yhteenveto siitä, kuinka monta sekuntia 
yksittäisen kategoriaan kuuluvan testin (2000 polunetsintäongelmaa) 
keskimäärin vei. Yksittäisten testien mittaustulokset ja laskuihin käytetty 
R-ohjelma on saatavilla testiohjelman data-kansiosta. (\cite{gt2}) \par
	Tulkoksista näkyy, että ajoajat kasvavat kaikkialla selkeästi 
nopeammin kuin graafikoko, paitsi listatyyppisten graafien BFS:ssä. 
Matriisityyppisissä graafeissa Dijkstran algoritmi on vähän nopeampi kuin 
BFS, mutta listatyyppisissä graafeissa BFS on selkeästi nopeampi kuin 
Dijkstran algoritmi. Muissa vastaavissa tutkimuksissa, joita luin tätä 
tutkielmaa varten Dijkstra oli nopeampi kuin BFS.\cite{mazeGameTrilogi} 
Vaikka Dijkstran algoritmi onkin epäinformoitu hakualgoritmi\cite{arXivMAPF}, 
se käyttää ahneuden periaatetta käsiteltävien solmujen ja kaarien 
minimoimiseksi\cite{mazeGameTrilogi}, jonka takia sen hitaus listatyyppisissä 
graafeissa BFS:ään verrattuna oli yllättävää. Mahdollinen selitys tälle voi 
ehkä olla testiohjelman toteutus Dijkstran algoritmista, joka iteroi 
tutkittavasta solmusta lähtevien kaarien läpi.\cite{gt2} Tämä saattaa olla 
listatyyppisissä graafeissa kallis toimenpide.
