\chapter{Eräiden algoritmien tehokkuuden tarkastelu esimerkkiongelmassa}\label{benchmarking}

Tässä tutkielmassa on verrattu kahta polunetsintäalgoritmia, Dijkstran 
algoritmia ja BFS:ää, ajamalla niitä koneellisesti monta kertaa satunnaisesti 
generoiduissa graafeissa ja tutkimalla ajautumisaikoja. Graafit on toteutettu
Boost Graph Library C++ kirjaston avulla jonka kehitti \textcite{bgl} . 
Testiohjelma ajaa algoritmit kahdessa Boost Graph Libraryn eri 
graafitoteutuksessa. Tämän tarkoitus on demonstroida graafien toteutuksen 
vaikutusta algoritmeihin. Testiohjelma itsessään on kehitetty tätä 
projektia varten ja julkaistu kokonaisuudessaan avoimen lähdekoodin jakeluun.
\cite{gt2} \par
	Testiohjelma ajaa jokaisella ajokerralla 2000 testiä, jossa jokaista 
varten generoidaan graafi $n$ solmulla ja $1,25n$ kaarella. Käytetyt 
graafikoot olivat $n=64$, $n=128$ ja $n=512$. Toteutuksen testisilmukassa 
oli kutsu, sekä BFS:lle, että Dijkstran algoritmille, sekä listatyyppisessä-, 
että matriisityyppisessä graafissa, mutta yksittäisissä testeissä 
kommentoitiin kaikki muu kuin testattava. Täten kun testattiin esimerkiksi 
Dijkstran algoritmia matriisityyppisessä graafissa, niin kaikki 
listatyyppisiin graafeihin ja BFS:ään liittyvät toimenpiteet oli kommentoitu 
pois, jolloin ne eivät vaikuta testien tuloksiin.\cite{gt2} \par

\begin{table}
\centering{}\caption{Yhteenveto testeissä mitatuista ajoista\label{tab:my-table1}}
\begin{tabular}{l|c|c|}
Graafien tyyppi ja koko (solmuja) & BFS & Dijkstran algoritmi \tabularnewline
\hline 
				Lista 64 	& 
\begin{tabular}{@{}c@{}}	Keskiarvo: 2,805 \\ Keskihajonta: 0,3972	\end{tabular} & 
\begin{tabular}{@{}c@{}}	Keskiarvo: 2,975 \\ Keskihajonta: 0,1565	\end{tabular} \tabularnewline
\hline
				Lista 128 	& 
\begin{tabular}{@{}c@{}}	Keskiarvo: 5,345 \\ Keskihajonta: 0,4766	\end{tabular} & 
\begin{tabular}{@{}c@{}}	Keskiarvo: 6,63  \\ Keskihajonta: 0,4840	\end{tabular} \tabularnewline
\hline
				Lista 512 	& 
\begin{tabular}{@{}c@{}}	Keskiarvo: 25,67 \\ Keskihajonta: 0,4714	\end{tabular} & 
\begin{tabular}{@{}c@{}}	Keskiarvo: 50,4  \\ Keskihajonta: 0,5931	\end{tabular} \tabularnewline
\hline
				Matriisi 64 	& 
\begin{tabular}{@{}c@{}}	Keskiarvo: 3,185 \\ Keskihajonta: 0,3893	\end{tabular} & 
\begin{tabular}{@{}c@{}}	Keskiarvo: 2,945 \\ Keskihajonta: 0,2286	\end{tabular} \tabularnewline
\hline 
				Matriisi 128 	& 
\begin{tabular}{@{}c@{}}	Keskiarvo: 9,49  \\ Keskihajonta: 0,5012	\end{tabular} & 
\begin{tabular}{@{}c@{}}	Keskiarvo: 8,77  \\ Keskihajonta: 0,4219	\end{tabular} \tabularnewline
\hline 
				Matriisi 512 	& 
\begin{tabular}{@{}c@{}}	Keskiarvo: 124,73 \\ Keskihajonta: 0,6156	\end{tabular} & 
\begin{tabular}{@{}c@{}}	Keskiarvo: 114,315 \\ Keskihajonta: 1,214	\end{tabular} \tabularnewline
\end{tabular}
\end{table}

Väliaikaista tekstiä, jotta näen miltä tulostaulukko näyttää takstin ympäröimänä:
Dipi diipa diipa doudou. Diipaa didi dou. Tyryry ryryryryry ryryry ryryry ry ryryryry turururu tururuu tu.
Dipi diipa diipa doudou. Diipaa didi dou. Tyryry ryryryryry ryryry ryryry ry ryryryry turururu tururuu tu.
Dipi diipa diipa doudou. Diipaa didi dou. Tyryry ryryryryry ryryry ryryry ry ryryryry turururu tururuu tu.
Dipi diipa diipa doudou. Diipaa didi dou. Tyryry ryryryryry ryryry ryryry ry ryryryry turururu tururuu tu.
Dipi diipa diipa doudou. Diipaa didi dou. Tyryry ryryryryry ryryry ryryry ry ryryryry turururu tururuu tu.
\par

Dipi diipa diipa doudou. Diipaa didi dou. Tyryry ryryryryry ryryry ryryry ry ryryryry turururu tururuu tu.
Dipi diipa diipa doudou. Diipaa didi dou. Tyryry ryryryryry ryryry ryryry ry ryryryry turururu tururuu tu.
Dipi diipa diipa doudou. Diipaa didi dou. Tyryry ryryryryry ryryry ryryry ry ryryryry turururu tururuu tu.
Dipi diipa diipa doudou. Diipaa didi dou. Tyryry ryryryryry ryryry ryryry ry ryryryry turururu tururuu tu.
Dipi diipa diipa doudou. Diipaa didi dou. Tyryry ryryryryry ryryry ryryry ry ryryryry turururu tururuu tu.

