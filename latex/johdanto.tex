\chapter{Johdanto} \label{Johdanto}

\section{Tutkielman tarkoitus}\label{tTarkoitus}
Polunetsintäalgoritmeja tarvitaan aina kun halutaan koneellisesti etsiä 
polku kahden pisteen väliin. Ne ovat paljon tutkittu aihepiiri matematiikan, 
tietotekniikan, algoritmiikan ja tekoälytutkimuksen alalla. Tämän tutkielman 
tarkoitus on syventyä aihepiiriin ja koota sen perusteet yhteen niin, että sen 
lukemisen jälkeen lukijalla on perusymmärrys aihepiiristä. Tutkielma 
tarkastelee erilaisia polunetsintäalgoritmeja, sekä vertailee niiden 
toimintaa ja tehokkuutta esimerkkien avulla.

\section{Tutkimuskysymykset}\label{tutkimuskysymykset}
Tutkielmassa pyritään vastaamaan seuraaviin kysymyksiin:
\begin{enumerate}[label=\textbf{\arabic*.}]
	\item \label{tKysymys1} \textbf{Tutkimuskysymys:} Minkälaisia polunetsintäalgoritmeja on kehitetty?
	\item \label{tKysymys2} \textbf{Tutkimuskysymys:} Miten niitä voidaan käyttää käytännön sovelluksiin?
	\item \label{tKysymys3} \textbf{Tutkimuskysymys:} Miten niiden tehokkuutta voidaan mitata?
\end{enumerate}

\section{Tiedonhakumenetelmät}\label{tiedonhakuM}
Tietoa tämän tutkielman tekoon on haettu IEEE:n Xplore Digital 
Center-tietokannasta, Web of Science-tietokannasta, sekä Google 
Scholar-hakupalvelusta. Hakutuloksia rajattiin julkaisuajan mukaan niin, 
että suurin osa hakutuloksista on julkaistu vuonna 2018 tai sen jälkeen. 
Myös aihepiirirajausta on käytetty. Hakusanoissa on käytetty osuvempien 
tulosten löytämiseksi Boolen operaattoreita sekä sanakatkaisua. Alla on 
muutama esimerkki käytetyistä hakusanoista:
\begin{center}
\texttt{
	pathfinding AND (grid based OR graph theory) AND "map*"	\\
	"pathfinding" AND "video gam*"				\\
	comparing AND "pathfinding algorithms"			\\
}
\end{center}

\section{Tutkielman rakenne}\label{tRakenne}
Tutkielman luku \ref{Taustoitus} taustoittaa myöhempiä lukuja. Luku, 
\ref{Taustoitus} esittelee polunetsintään liittyvät peruskäsitteet ja 
tausta-aiheet, jotta seuraavien lukujen ymmärtäminen helpottuisi. 
Luvussa \ref{joitainP} käydään läpi muutaman tunnetun polunetsintäalgoritmin 
toiminta ja täten pyritään vastaamaan tutkimuskysymykseen \ref{tKysymys1} 
Luvussa \ref{algoritmienSovelluskohteita} käydään läpi joitakin 
polunetsintäalgoritmien yleisiä käyttökohteita ja pyritään vastaamaan 
tutkimuskysymykseen \ref{tKysymys2} Luvussa \ref{benchmarking} taas mitataan 
useiden eri polunetsintäalgoritmien tehokkuus eräässä esimerkkiongelmassa ja 
vertaillaan niitä tämän avulla toisiinsa. Lopussa olevassa 
yhteenvetokappaleessa \ref{yhteenveto} tulokset kootaan vielä yhteen ja 
kerrataan tärkeimmät havainnot lyhyesti.
