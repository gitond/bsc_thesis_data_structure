\chapter{Joitain polunetsintäalgoritmeja}\label{joitainP}

\section{Leveyssuuntainen läpikäynti (BFS)}\label{bfs}
Leveyssuuntainen läpikäynti, eli leveyshaku (Breadth First Search, BFS) on 
epäinformoitu hakualgoritmi, joka perustuu sokeaan 
hakuun.\cite{applSciLawande} Siinä graafin solmut ryhmitellään eri tasoihin 
sen mukaan monenko kaaren kautta pitää kulkea lähtösolmusta, jotta niihin 
päästään. Lähtösolmu on siis tasolla 0, siihen yhdistyneet solmut tasolla 1, 
tason 1 solmuihin yhdistyneet solmut tasolla 2 ja niin edelleen. 
Leveyshaussa graafin kaikki solmut käydään läpi niin, että tarkistetaan 
onko solmussa jo käyty, onko solmu maalisolmu ja mihin solmuihin sillä on 
yhteys. Sitten tallennetaan solmu läpikäytyjen solmujen listalle ja tieto 
siitä, mitä kautta solmulle ollaan tultu.

\section{Syvyyssuuntainen läpikäynti (DFS)}\label{dfs}
\section{Dijkstran algoritmi}\label{dijkstra}
\section{A*-algoritmi}\label{aStar}
\section{Hierarkinen polunetsintä}\label{hpa}
