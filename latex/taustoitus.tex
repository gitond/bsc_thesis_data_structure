\chapter{Taustoitus} \label{Taustoitus}

\section{Polunetsintä ongelmana}\label{pOngelmana}
Polunetsintä tarkoittaa tietotekniikan kontekstissa ongelmaa, jossa halutaan 
koneellisesti löytää, sekä mahdollisesti myös piirtää sallittu polku kahden 
etukäteen määritellyn pisteen väliin. Useissa polunetsinnän liittyvissä 
ongelmissa halutaan myös, että löydetty polku olisi jollain tavalla 
optimaalinen. Optimaalinen polku voisi esimerkiksi olla lyhyempi tai nopeampi 
kuin muut mahdolliset polut. \cite{MathewAndMalathy} \par
	Jotta polunetsintäongelmia voitaisiin ratkaista koneellisesti, 
tarvitsee ne ensin muuttaa matemaattiseen esitysmuotoon. Tämän vuoksi 
polunetsintäongelmat voidaan jakaa kahteen osaonglemaan: graafigeneraatioon 
ja polunetsintäalgoritmin käyttöön. Graafigeneraatiota tarvitaan, koska polun 
löytämiseen käytetyt algoritmit toimivat tyypillisesti graafeissa. Siinä 
muutetaan polunetsintäongelman alueena toimiva maasto- tai kartta-alue 
graafimuotoiseksi rakentamalla siitä esimerkiksi luurankomalli 
(skeletonization). \cite{ACMHindawi} Tämä tutkielma on keskittynyt 
polunetsintäalgoritmeihin ja niiden käyttöön, eikä graafigeneraatiota tämän 
vuoksi käsitellä tutkielmassa laajemmin.

\\
\paragraph{\textit{* Tässä voisi olla jokin kuva kartan muuttamisesta graafiksi *}}\mbox{}
\\

	Polunetsintäongelmat voidaan myös jakaa yksiagentillisiin 
(single-agent) ja moniagentillisiin (multi-agent) polunetsintäongelmiin. 
Yksiagentillisissa ongelmissa ongelma-alueella liikkuu vain yksi agentti, 
jolle pyritään löytämään optimaalinen polku jostain lähtöpisteestä johonkin 
päämäärään. Moniagentillisisssa ongelmissa alueella liikkuvia agentteja on 
useita. Näissä kaikilla agenteilla on oma lähtöpiste ja oma päämäärä ja 
kaikille tulee löytää optimaaliset polut niin, että agentit eivät törmäile 
toisiinsa ja niiden välille ei synny reititykseen liittyviä konflikteja. 
\cite{arXivMAPF} Näistä ongelmatyypeistä tämä tutkielma on käsittelee 
pääasiassa yksiagentilliseen polunetsintään kehitettyjä 
polunetsintäalgoritmeja.

\section{Algoritmeista}\label{algoritmeista}
\section{Esimerkkejä sovelluskohteista}\label{eSuovelluskohteista}
