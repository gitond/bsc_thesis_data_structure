\chapter{Taustoitus} \label{Taustoitus}

\section{Polunetsintä ongelmana}\label{pOngelmana}
Polunetsintä tarkoittaa tietotekniikan kontekstissa ongelmaa, jossa halutaan 
koneellisesti löytää, sekä mahdollisesti myös piirtää polku kahden 
etukäteen määritellyn pisteen väliin. Useissa polunetsinnän liittyvissä 
ongelmissa halutaan myös, että löydetty polku olisi jollain tavalla 
optimaalinen. Optimaalinen polku voisi esimerkiksi olla lyhyempi tai nopeampi 
kuin muut mahdolliset polut\cite{MathewAndMalathy}.\par
	Jotta polunetsintäongelmia voitaisiin ratkaista koneellisesti, 
tarvitsee ne ensin muuttaa matemaattiseen esitysmuotoon. Tämän vuoksi 
polunetsintäongelmat voidaan jakaa kahteen osaongelmaan: graafigeneraatioon 
ja polunetsintäalgoritmin käyttöön. Graafigeneraatiota tarvitaan, koska polun 
löytämiseen käytetyt algoritmit toimivat tyypillisesti graafeissa. Siinä 
muutetaan polunetsintäongelman alueena toimiva maasto- tai kartta-alue 
graafimuotoiseksi rakentamalla siitä esimerkiksi luurankomalli 
(skeletonization)\cite{ACMHindawi}. Tämä tutkielma on keskittynyt 
jälkimmäiseen ongelmaan, polunetsintäalgoritmeihin ja niiden käyttöön, eikä 
graafigeneraatiota tämän vuoksi käsitellä tutkielmassa laajemmin. \par
	Polunetsintäongelmat voidaan edelleen jakaa yksiagentillisiin 
(single-agent) ja moniagentillisiin (multi-agent) polunetsintäongelmiin. 
Yksiagentillisissa ongelmissa ongelma-alueella liikkuu vain yksi agentti, 
jolle pyritään löytämään optimaalinen polku jostain lähtöpisteestä johonkin 
päämäärään. Moniagentillisissa ongelmissa alueella liikkuvia agentteja on 
useita. Näissä kaikilla agenteilla on oma lähtöpiste ja oma päämäärä ja 
kaikille tulee löytää optimaaliset polut niin, että agentit eivät törmäile 
toisiinsa ja niiden välille ei synny reititykseen liittyviä 
konflikteja\cite{arXivMAPF}. Näistä ongelmatyypeistä tämä tutkielma käsittelee 
pääasiassa yksiagentilliseen polunetsintään kehitettyjä 
polunetsintäalgoritmeja.

\section{Algoritmeista}\label{algoritmeista}
Tässä tutkielmassa tarkasteltavat algoritmit ovat kaksiulotteisissa 
ajoaikana muuttumattomissa graafeissa toimivia yksiagenttisia 
polunetsintäalgoritmeja, ellei toisin mainita. Nämä algoritmit voidaan jakaa 
niiden toimintaperiaatteen mukaan epäinformoituihin (uninformed), 
informoituihin (informed) ja metaheuristisiin (metaheuristic) 
hakualgoritmeihin (search algorithms)\cite{applSciLawande}.\par
	Epäinformoidut hakualgoritmit ovat yksinkertaisia algoritmeja, jotka 
eivät ole tietoisia ongelma-alueensa yksityiskohdista. Näin ollen 
epäinformoidut hakualgoritmit perustuvat toimintamalliin, jossa kuljetaan 
graafissa solmukohdalta toiseen kaaria pitkin niin kauan kunnes ollaan 
löydetty polku lähtösolmusta maalisolmuun. Tätä toimintamallia sanotaan 
joskus myös sokean haun konseptiksi (blind search)\cite{applSciLawande}.\par
	Informoidut hakualgoritmit sen sijaan käyttävät ongelma-alueesta 
laskettuja tietoja hyväksi nopeuttaakseen ajoaikaa. Tyypillisesti tämä 
tehdään laskemalla seuraavaksi läpikäytävien solmukohtien etäisyys 
maalisolmusta käyttämällä niin kutsuttua heuristista funktiota. Näin ollen 
jokaiselle graafissa olevalle solmukohdalle $n$ saadaan laskettua sen 
kautta kulkevan polun hinta $F(n)$ käyttämällä hintafunktiota 
$F(n) = G(n) + H(n)$ , jossa $G(n)$ on solmulle $n$ asti kuljettu matka 
lähtösolmusta $H(n)$ on heuristisen funktion palauttama arvo. 
Korkeahintaisia solmukohtia ei tutkita algoritmin ajon aikana, jonka takia tutkittavia solmuja on yhteensä vähemmän ja algoritmi on 
nopeampi ajaa\cite{applSciLawande}.\par
	A* (luetaan A-tähti tai A-star) on yksi esimerkki informoidusta 
hakualgoritmista. A* on yksi suosituimmista polunetsintäalgoritmeista 
käytännön sovelluskohteista. Tämä johtuu siitä, että A* on yksinkertainen 
toteuttaa ja se palauttaa aina optimaalisen polun, mikäli käytetään 
sopivaa heuristista funktiota\cite{MathewAndMalathy}. Sen pohjalta on 
myös kehitetty muita polunetsintäalgoritmeja\cite{applSciLawande}.\par
	Metaheuristiset algoritmit eivät perustu heuristisiin funktioihin, tai 
solmukohtien tutkimiseen, vaan ne käyttävät muunlaisia keinoja polkujen 
etsimiseen\cite{applSciLawande}. Niitä ei käydä tässä tutkielmassa tarkemmin 
läpi.

\section{Esimerkkejä sovelluskohteista}\label{eSuovelluskohteista}
Polunetsintäongelmia joudutaan ratkomaan muun muassa videopeleissä, 
karttaohjelmissa, erilaisissa simulaatioissa ja robotiikan 
alalla\cite{ACMHindawi}, sekä esimerkiksi logistiikan alan automaatiossa ja 
robotisoitujen autojen kehityksessä\cite{arXivMAPF}. Merkittävän suuri 
osa polunetsintäongelmiin liittyvistä tutkimuksista tehdään nimenomaan 
videopelien\cite{MathewAndMalathy}\cite{ACMHindawi}\cite{mazeGameTrilogi}
ja robotiikan\cite{ACMHindawi}\cite{DelaunayVoronoiAStar} näkökulmasta.\par
	Polunetsintäongelmia esiintyy videopeleissä, koska niissä on usein 
tarve simuloida ei-pelattavien hahmojen (non-playable character, NPC) 
liikkeitä niin, että niille on määritelty lähtö- ja maalipisteet, sekä 
sallitut ja kielletyt liikkumisalueet. Kun polunetsintäalgoritmeja 
sovelletaan videopeleihin, on tärkeää, että algoritmit olisivat 
laskennallisesti tehokkaita. Esimerkiksi reaaliaikaisissa strategiapeleissä 
(real-time strategy, RTS) pelaaja liikuttaa useita eri yksiköitä eri puolille 
karttaa merkitsemällä niille pisteitä joiden kautta kulkea. Näissä 
tilanteissa on pelikokemuksen kannalta tärkeää, ettei eri yksiköille 
suoritettavat polunetsintäalgoritmit vaikuttaisi pelin 
sulavuuteen\cite{MathewAndMalathy}. \textcite{mazeGameTrilogi} tutkivat 
eri polunetsintäalgoritmien toimimista videopelissä, jossa pelaaja rakentaa 
valmiiksi annetuista palikoista labyrintin ja algoritmi yrittää ratkaista 
sen. Tutkimuksessa A* osoittautui tutkituista algoritmeista parhaaksi. 
Vastaavankaltainen tutkimus käydään yksityiskohtaisemmin läpi tutkielman 
luvussa \ref{benchmarking}. \par
	Videopelien lisäksi polunetsintäalgoritmeja sovelletaan myös 
karttaohjelmissa. Niissä polunetsintää käytetään reitinhakuun, joka on 
karttaohjelmien eräs päätoiminnallisuus. Reitinhakuun käytetään usein 
Dijkstran algoritmia\cite{IOPDijkstra}, josta puhutaan tarkemmin tutkielman 
luvussa \ref{dijkstra}. Karttaohjelmissa polunetsintä on kuitenkin melko 
erilaista kuin videopeleissä. On tärkeää, että yksittäisen optimaalisen reitin 
sijasta reitinhaku palauttaisi käyttäjälle useita vaihtoehtoisia ajoreittejä, 
joita generoitaisiin lisää ajon aikana siltä varalta, että liikenteen 
olosuhteet muuttuvat\cite{Lanelet2}. \textcite{Lanelet2} käyvät artikkelissaan 
läpi heidän suunnittelemansa digitaalisen kartta-alustan Lanelet2:n 
lähestymistapaa reitinhakuun. Polunetsintäalgoritmien käyttö karttaohjelmien 
reitinhaussa käydään tarkemmin läpi tutkielman luvussa 
\ref{karttaohjelmat}.\par
	Polunetsintää on tutkittu myös paljon robotiikan näkökulmasta. 
Robotiikassa polunetsintää joudutaan soveltamaan muun muassa kun 
kehitetään itseajavia ajoneuvoja ja tehtaassa liikkuvia 
teollisuusrobotteja\cite{arXivMAPF}. Esimerkiksi \textcite{Lanelet2} 
pohtivat artikkelissaan muun muassa sitä, miten itseajaville autoille 
voitaisiin kehittää karttapalveluita, joita ne voisivat soveltaa. 
Polunetsinnän soveltaminen robotiikassa on hyvin monipuolista, koska 
ongelmat ovat keskenään erilaisia ja lähestymistavatkin voivat olla 
erilaisia. Esimerkiksi jos polunetsintää käytetään tilanteessa, jossa 
tehtaassa liikkuu kymmeniä teollisuusrobotteja eri kohteisiin, niin 
polunetsintäalgoritmeja voidaan ajaa jokaisessa robotissa erikseen, tai 
keskustietokone voi välittää jokaiselle robotille polun, jotka muodostavat 
ratkaisun moniagentilliseen polunetsintäongelmaan. 
\textcite{DelaunayVoronoiAStar} esittävät artikkelissaan 
Delaunay-kolmiomittaukseen perustuvaa graafigeneraatiota ja tehostetun 
A*-algoritmin ajoa liikkuvassa robotissa. Polunetsintäalgoritmien 
soveltamista robotiikan sovelluksiin käydään tarkemmin läpi tutkielman 
luvussa \ref{robotiikka}.
