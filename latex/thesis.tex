% Document template suitable for use as a LaTeX master-file 
% for thesis works in University of Turku Department of Computing
%
% Technical usage guide: https://tech.utugit.fi/soft/thesis/doc/doc/overview/
% 

\documentclass[language=finnish,version=final,mainfont=none,sharelatex=false]{utuftthesis}
\setcounter{secnumdepth}{2}
\setcounter{tocdepth}{2}
\usepackage{float}
\usepackage{enumitem}
\usepackage[caption=false]{subfig}

% Define the algorithm environment
%\makeatletter
\providecommand\textquotedblplain{%
  \bgroup\addfontfeatures{Mapping=}\char34\egroup}
\providecommand{\tabularnewline}{\\}
\floatstyle{ruled}
\newfloat{algorithm}{tbp}{loa}
\providecommand{\algorithmname}{Algoritmi}
\floatname{algorithm}{\protect\algorithmname}
%\makeatother

\addbibresource{Bibliografia.bib}

\begin{document}

\pubyear{2023}
\pubmonth{2}
\publab{Labran nimi}
\publaben{Laboratory Name}
\pubtype{tkk}
\title{Polunetsintäalgoritmit (väliaikainen otsikko)}
\author{Botond Ortutay}

\maketitle
\keywords{algoritmiikka, polunetsintä, graafiteoria}

\keywordsen{algorithmics, path finding, graph theory}
\begin{abstract}
\textit{Tässä tutkielmassa tarkastellaan polunetsintäalgoritmeja koneellisen 
polunetsinnän näkökulmasta. Tutkielma on kirjallisuuskatsaus, eli se perustuu 
pääosin aiheesta julkaistuun tieteelliseen kirjallisuuteen, mutta se sisältää 
pienimuotoisen tutkimuksen eräiden algoritmien mittaamisesta 
testiympäristössä. Tutkielma esittellee lukijalle joitakin 
polunetsintäalgoritmeja, sekä niiden käyttökohteita ja vertailee niiden 
tehokkuutta.}
\end{abstract}

\begin{abstracten}
\textit{This thesis examines path finding algorithms from a computing 
oriented point of view. It is a literature review, so it is mostly based on 
already published research. It also contains a small set of performance 
measurements of certain algorithms in a test environment. The thesis 
introduces the reader to a few path finding algorithms, their use cases and 
it compares their performance.}
\end{abstracten}



% mandatory
\tableofcontents

% if you want a list of figures
%\listoffigures

% if you want a list of tables
\listoftables

% if you want a list of acronyms
%\listofacronyms

% change the name if the default doesn't sound right
\renewcommand{\algorithmname}{\listingscaption}

% The thesis starts here.

\begin{comment}
To better organize things, create a new tex file for each chapter
and input it below.

Avoid using the å, ä, ö or <space> characters in referred names and
underscores \_ in file names (may break hyperref).

Good luck!
\end{comment}

\chapter{Johdanto} \label{Johdanto}

\section{Tutkielman tarkoitus}\label{tTarkoitus}
\begingroup
\itshape
\paragraph{*Suunnitelma kappaleelle 1.1:*}
\begin{itemize}
	\item Harkittu ja kiinnostava aloitus
	\item Käy lyhyesti ja yksinkertaisesti läpi seuraavat asiat:
	\begin{itemize}
		\item Polunetsintäalgoritmejä tarvitaan kun...
		\item Polunetsintä käsitteenä
		\item Miski kirjoitin kandityön juuri tästä aiheesta? (tutkielman perustelu)
	\end{itemize}
	\item Päätä kappale jotenkin näin:
\end{itemize}
"Tutkielman tarkoitus on esitellä lukijalle erilaisia polunetsintäalgoritmeja, 
sekä verrata niiden toimintaa jossakin esimerkkiympäristössä"
\endgroup

\section{Tutkimuskysymykset}\label{tutkimuskysymykset}
Tutkielmassa pyritöön vastaamaan seuraaviin kysymyksiin:
\begin{enumerate}[label=\textbf{\arabic*.}]
	\item \label{tKysymys1} \textbf{Tutkimuskysymys:} Minkälaisia polunetsintäalgoritmeja on kehitetty?
	\item \label{tKysymys2} \textbf{Tutkimuskysymys:} Miten niitä voidaan käyttää käytännön sovelluksiin?
	\item \label{tKysymys3} \textbf{Tutkimuskysymys:} Miten niiden tehokkuutta voidaan mitata?
\end{enumerate}

\section{Tiedonhakumenetelmät}\label{tiedonhakuM}
Tietoa tämän tutkielman tekoon on haettu IEEE:n Xplore Digital 
Center-tietokannasta, Web of Science-tietokannasta, sekä Google 
Scholar-hakupalvelusta. Hakutuloksia rajattiin julkaisuajan mukaan niin, 
että suurin osa hakutuloksista on julkaistu vuona 2018 tai sen jälkeen. 
Myös aihepiirirajausta on käytetty. Hakusanoissa on käytetty osuvempien 
tulosten löytämiseksi Boolen operaattoreita, sekä sanakatkaisua. Alla on 
muutama esimerkki käytetyistä hakusanoista:
\begin{center}
\texttt{
	pathfinding AND (grid based OR graph theory) AND "map*"	\\
	"pathfinding" AND "video gam*"				\\
	comparing AND "pathfinding algorithms"			\\
}
\end{center}

\section{Tutkielman rakenne}\label{tRakenne}
Tutkielman luku \ref{Taustoitus} taustoittaa seuraavia lukuja. Tarkoitus on, 
että luvun \ref{Taustoitus} lukemisen jälkeen lukijalle tulisivat tutuksi 
polunetsintään liittyvät peruskäsitteet ja tausta-aiheet, jotta seuraavien 
lukujen ymmärtäminen helpottuisi. Luvussa \ref{joitainP} käydään läpi 
muutaman tunnetun polunetsinnän toiminta ja täten pyritään vastaamaan 
tutkimuskysymykseen \ref{tKysymys1} Luvussa 
\ref{algoritmienSovelluskohteita} käydään läpi joitakin 
polunetsintäaldoritmien yleisiä käyttökohteita ja pyritään vastaamaan 
tutkimuskysymykseen \ref{tKysymys2} Luvussa \ref{benchmarking} taas mitataan 
useiden eri polunetsintäalgoritmien tehokkuus eräässä esimerkkiongelmassa ja 
vertaillaan niitä tämän avulla toisiinsa. Lopussa olevassa 
yhteenvetokappaleessa \ref{yhteenveto} tulokset kootaan vielä yhteen ja 
esitetään helpommin luettavassa muodossa.

\chapter{Taustoitus} \label{Taustoitus}
\section{Polunetsintä ongelmana}\label{pOngelmana}
\section{Algoritmeista}\label{algoritmeista}
\section{Esimerkkejä sovelluskohteista}\label{eSuovelluskohteista}

\chapter{Joitain polunetsintäalgoritmeja}\label{joitainP}

\section{Leveyssuuntainen läpikäynti (BFS)}\label{bfs}
Leveyssuuntainen läpikäynti, eli leveyshaku (Breadth First Search, BFS) on 
epäinformoitu hakualgoritmi, joka perustuu sokeaan 
hakuun.\cite{applSciLawande} Siinä graafin solmut ryhmitellään eri tasoihin 
sen mukaan monenko kaaren kautta pitää kulkea lähtösolmusta, jotta niihin 
päästään. Lähtösolmu on siis tasolla 0, siihen yhdistyneet solmut tasolla 1, 
tason 1 solmuihin yhdistyneet solmut tasolla 2 ja niin edelleen. 
Leveyshaussa graafin kaikki solmut käydään läpi niin, että tarkistetaan 
onko solmussa jo käyty, onko solmu maalisolmu ja mihin solmuihin sillä on 
yhteys. Sitten tallennetaan solmu läpikäytyjen solmujen listalle ja tieto 
siitä, mitä kautta solmulle ollaan tultu.\cite{BFSRahim} \par
	BFS:n etuihin kuuluu, että tällä algoritmilla voidaan löytää kaikki 
mahdolliset polut lähtösolmusta maalisolmuun, koska se käy läpi kaikki graafin 
solmut.\cite{BFSRahim} Näistä voidaan sitten valita optimaalisin polku.
Haittapuoliin kuuluu suuri muistinkulutus tallennettujen polkujen 
lukumäärän takia,\cite{BFSRahim} sekä pitkä ajoaika.\cite{mazeGameTrilogi} 
Alla \textcite{applSciLawande} perusteella kirjoitettu pseudokoodi 
BFS-algoritmille:
\begin{algorithm}[H]
\caption{Esimerkki BFS-algoritmista}\label{BFSEsim}
\begin{algoritmic}
\Procedure{BFSEsimerkki}{$graafidata,lahtosolmu,maalisolmu$}
	\State $vieraillut \gets tyhjaLista$
	\State $kasiteltavat \gets tyhjaLista$
	\State $kasiteltavat.lisaaLoppuun(nykyinen=lahtosolmu;vanhempi=NULL)$
	\While{$kasiteltavat\not=tyhjaLista$}
		\State $kasiteltava \gets kasiteltavat.poistaEnsimmainen()$
		\State $vieraillut.lisaaLoppuun(kasiteltavat)$
		\If{$kasiteltava.nykyinen = maalisolmu$}
			\State $polku \gets tyhjaLista$
			\While{$kasiteltava.vanhempi\not=NULL$}
				\State $polku.lisaaAlkuun(kasiteltava.nykyinen)$
				\State $kasiteltava \gets kasiteltava.vanhempi$
			\EndWhile
			\State $polku.lisaaAlkuun(lahtosolmu)$
			\State \textbf{return} $polku$
		\Else
			\ForAll{$n \in kasiteltava.nykyinen.naapurit$}
				\State $(n \not\in kasitellyt ?  kasiteltavat.lisaaLoppuun$
				\State $(nykyinen = n;vanhempi = kasiteltava.nykyinen) :$
				\State \textbf{continue} $;$
			\EndFor
		\EndIf
	\EndWhile
\EndProcedure
\end{algoritmic}
\end{algorithm}

\section{Syvyyssuuntainen läpikäynti (DFS)}\label{dfs}
\section{Dijkstran algoritmi}\label{dijkstra}
\section{A*-algoritmi}\label{aStar}
\section{Hierarkinen polunetsintä}\label{hpa}

\chapter{Algoritmien sovelluskohteita} \label{algoritmienSovelluskohteita}

\section{Videopelit}\label{videopelit}
Monet nykyaikaiset polunetsintäalgoritmeihin liittyvät tutkimukset tehdään 
videopeliteollisuuden 
käyttöön.\cite{MathewAndMalathy}\cite{ACMHindawi}\cite{mazeGameTrilogi} 
Tämä johtuu siitä, että videopeleissä törmätään usein tilanteisiin, jossa 
NPC-hahmojen pitää liikkua videopelikartalla niin, että pelaaja ei suoraan 
itse ohjaa niitä. Tällöin pelin on itse löydettävä reitti kahden pisteen 
välille. Kun polunetsintäalgoritmeja toteutetaan videopeleissä, joudutaan 
kiinnittämään huomiota muutamiin erityisvaatimuksiin. Polunetsintäalgoritmien 
ajaminen ei saa olla liian resurssi-intensiivistä, eli sen pitää pyrkiä 
kuluttamaan tietokoneen muistia ja suoritinaikaa mahdollisimman vähän. Jotta 
resurssi-intensiivisyydeltä vältyttäisiin, voidaan löystää polun 
optimaalisuuteen liittyviä vaatimuksia, jolloin polun ei tarvitse olla enää 
lyhin mahdollinen.\cite{MathewAndMalathy} \par
	Esimerkiksi reaaliaikaisissa strategiapeleissä (real-time strategy, 
RTS) pelaaja komentaa usein eri yksiköitä eri puolelle pelialuetta 
mekitsemällä niille maalipisteitä. Näitä yksiköitä voi olla jopa satoja, 
joille pitää löytää polut suurella etsintä-alueella huomioiden mahdollisesti 
muut alueella liikkuvat yksiköt, sekä pelin säännöt siitä, että miten eri 
yksiköt käytännössä voivat liikkua.\cite{pPacman}\cite{MathewAndMalathy} 
Nämä monimutkaiset laskutoimitukset on suoritettava pelin logiikan 
vaatimalla nopeudella. Esimerkiksi videopeliyhtiö BioWare on asettanut 
säännön, jonka mukaan kaikkien pelialueella liikkuvien agenttien polunetsintä 
on kyttävä suorittamaan 1-3 ms aikana.\cite{pPacman} \par
	\textcite{pPacman} kehittivät Pathfinding-in-Pacman-projektin, jossa 
he sovelsivat polunetsintäalgoritmeja Pac-Manin kaltaiseen peliin. Pelissä 
pelihahmo liikkuu labyrintissä yrittäen kerätä pisteitä, samalla kun sitä 
jahtaavat vihollishahmoina toimivat haamut. Pelaaja voittaa kerättyään kaikki 
pisteet, ja häviää kun haamut koskevat häntä. Peli sisältää toteutukset BFS- 
ja A*-algoritmeista, sekä pelin tarkoituksiin muokatusta A*-algoritmistä, 
jota projektin raportti kutsuu Context Dependent Subgoaling A*-algoritmiksi
(kontekstiriippuvaisesti osatavoitteistava A*, CDSA*) ja ohjelmakoodi kutsuu 
subGoalAStar-algoritmiksi. CSDA* etsii polun haamulta pelaajahahmolle 
A*-algoritmilla, mikäli haamu on tarpeeksi lähellä pelaajahahmoa. Muussa 
tapauksessa CSDA* palauttaa ainoastaan suunnan, jolla haamu pääsee lähemmäs 
pelaajahahmoa. Tällä tavoin säästetään järjestelmän resursseja rajoittamalla 
tehtävien laskujen määrää. Pelissä käytetään polunetsintäalgoritmeja 
haamujen ja pelaajahahmojen välisen polun löytämiseksi ja A*- ja 
CSDA*-algoritmit käyttävät heuristiikkana Manhattan-etäisyyttä. \par
	\textcite{SturtevantDAO} vertaavat tutkimuksessaan erilaisia 
suurelle etsintäalueelle soveltuvia polunetsintätekniikoita Dragon Age: 
Origins-videopelissä (DAO) käytettyyn polunetsintään. Nämä 
polunetsintätekniikat ovat abstraktiahierarkien käyttö, parempien 
heurististen funktioiden käyttö, sekä sopimushierarkioiden (Contrarction 
Hierarchy, CH) käyttö. Abstraktihierarkia ja CH ovat hierarkisessa 
polunetsinnässä (luku \ref{hpa}) käytettäviä tapoja muodostaa ylätason 
graafi. Abstraktihierarkia muodostaa ylätason graafin niin, että 
varsinaisen etsintäalueen eri osa-alueet ovat ylätason graafin solmuja, kun 
taas CH muodostaa ylätason graafin laskemalla alkuperäisen graafin jokaiselle 
solmulle tärkeystason (importance level) ja poistamalla vähemmän tärkeät 
solmut graafista. \textcite{SturtevantDAO} osoittivat, että hierarkinen 
polunetsintä joko abstraktiohierarkiaa tai CH:ta käyttäen soveltui parhaiten 
polunetsintään DAO:n kaltaisissa videopeleissä, koska näillä menetelmillä 
käytettiin vähiten suoritinta ja muistia. DAO käyttää abstraktiohierarkiaa, 
mutta \textcite{SturtevantDAO} mainitsevat, että pelin polunetsintää olisi 
voinut tehostaa entisestään käyttämällä suurempia osa-alueita ylätason 
graafin rakentamiseen.

\section{Karttaohjelmat}\label{karttaohjelmat}
Polunetsintäalgoritmit ovat myös keskeisessä osassa karttaohjelmistoissa. 
Monet karttaohjelmat tarjoavat reitinhakupalveluita, joissa generoidaan kartan  
eri pisteiden välille reitti käyttäen polunetsintäalgoritmeja. Tämä on usein 
helppo toteuttaa, koska karttaohjelmat säilyttävät karttadataa usein 
polunetsintäalgoritmeille ideaalisessa muodossa. Esimerkiksi OpenDrive-
formaatti sisältää dataa risteyksistä ja teistä, josta kootaan 
reititysgraafi.\cite{Lanelet2} \par
	\textcite{OpenStreetMap} on yhteisön ylläpitämä avoimen lähdekoodin 
karttatietokanta, joka tarjoaa sivustollaan myös tietokannan karttadataa 
käyttävää karttaohjelmaa.\cite{OSMAbout} OpenStreetMap-karttaohjelma sisältää 
myös reitinhakumahdollisuuden. Reitinhakumahdollisuus sisältää 
reitinhakuvaihtoehdot kävelylle, pyöräilylle ja autoilulle käyttäen joko 
GraphHopperia, OSRM:ää tai Valhallaa.\cite{OpenStreetMap} \par
	Nämä ovat niin sanottuja reititysmoottoreita (routing engine), jotka 
ovat valmiita ohjelmistototeutuksia polunetsintäfunktioille.\cite{graphhopper} 
OpenStreetMapin käyttämä \textcite{graphhopper} on Javalla kirjoitettu 
reititysmoottori. Se käyttää sisäisessä toteutuksessaan Dijkstran algoritmia, 
A*-algoritmia, sekä hierarkista polunetsintää käyttäen CH-menetelmää ylätason 
graafin luomiseen ja Dijkstran algoritmiä polunetsintään. GraphHopper valitsee 
ongelmaan sopivan polunetsintämenetelmän sen asetusten perusteella.

\section{Robotiikka}\label{robotiikka}
Polunetsintää tutkitaan myös robotiikan näkökulmasta, koska käytännössä 
kaikki robotit, jotka liikkuvaat jossakin ympäristössä, tarvitsevat 
polunetsintäalgoritmeja tehdäkseen päätöksiä siitä, minne 
mennä.\cite{ProcediaAStar} Liikkuvien robottien polunetsintään liittyy 
monenlaisia haasteita. Yksi näistä on törmäyksien estäminen.\cite{ACMHindawi} 
Robotit voivat törmätessään aiheuttaa vakavaa vahinkoa itseensä, 
ympäristöönsä ja alueella liikkuviin ihmisiin.\cite{ProcediaAStar} Esimerkiksi 
itseajavien autojen tapauksessa törmäykset voidaan pyrkiä estämään 
huomioimalla muut tienkäyttäjät paikallisessa reitinsuunnittelussa. Tämä 
ongelma kuitenkin monimutkaistuu kun tarvitsee huomioida erilaisia 
tienkäyttäjiä. Muut tienkäyttäjät huomioiva paikallinen reitinhakualgoritmi on 
helpompi suunnitella moottoritielle, missä voidaan odottaa liikkuvan 
pelkästään autoja ja voidaan odottaa että kaikilla on tarpeeksi tilaa, kuin 
kaupunkiin, jossa on lisäksi jalankulkijoita ja pyöräilijöitä, sekä 
määrällisesti enemmän tienkäyttäjiä ja vähemmän tilaa.\cite{Lanelet2} \par
	Törmäysten estämisen lisäksi robotiikassa tärkeässä roolissa on myös 
graafigeneraatio. Graafigeneraatio on osa lähes jokaista polunetsintäongelmaa, 
koska etsintäalue pitää muuttaa graafiksi, jotta polunetsintäalgoritmia voisi 
käyttää. Graafigeneraatio kuitenkin korostuu robotiikan sovelluksissa, koska 
toisin kun esimerkiksi videopeleissä tai kartanohjelmissa, tietokone ei 
hallitse etsintäaluetta, eikä se välttämättä ole tietoinen esteiden 
sijainnista alueella, tai alueen muutoksesta. Tätä varten robotiikassa 
käytetään usein niin sanottuja dynaamisia polunetsintäalgoritmeja, jotka 
kykenevät reagoimaan etsintäalueen muutoksiin. Esimerkkejä näistä ovat 
A*-algoritmiin perustuva D*-algoritmi\cite{applSciLawande} ja 
\textcite{DelaunayVoronoiAStar} kehittämä DFPA-algoritmi. \par
	\textcite{transportRobotAStar} esittelevät artikkelissaan 
kehittämäänsä halpaa kuljetusrobottia, joka käyttää polunetsintään 
A*-algoritmia. Robotille syötetään ensin graafidata, jonka perusteella se 
sitten etsii lyhimmän polun A*-algoritmilla ja seuraa sitä maaliin asti. 
Robotti ei käytä dynaamista polunetsintää, eikä se kykene reagoimaan 
etsintäalueella tapahtuviin muutoksiin, vaan sille syötetään alueella olevat 
esteet graafidatan mukana. Lisäksi robotti sijoittaa itsensä graafiin 
seuraamalla toiminta-alueella maahan kiinnitettyjä mustia viivoja. Kyseinen 
robotti ei siis mielestäni soveltuisi tosielämän kuljetusrobotiksi, koska se 
ei pysty reagoimaan ympäristönsä muutoksiin ja se vaatii esiasennettua 
infrastruktuuria toimiakseen. Tutkimus keskittyi kuitenkin sensorisijoitusten 
optimalisointiin ja käännösmekanismin ja -algoritmien tekemiseen realistisen 
kuljetusrobotin polunetsintää enemmän. \par
	\textcite{warehouse} vertailevat artikkelissaan 
polunetsintäalgoritmeja varastoissa työskenteleville teollisuusroboteille. 
Koska varastossa saattaa liikkua kerralla jopa kymmeniä tai satoja robotteja, 
kyseessä on moniagentillinen polunetsintäongelma. Lisäksi artikkeli olettaa, 
että robotit liikkuvat varastossa jatkuvasti lastausalueiden ja 
varastointialueiden välillä niin, että kun jokin robotti saa sille asetetun 
tehtävän valmiiksi, sille annetaan uusi satunnainen tehtävä. \par
	Robotin sijaintia ei siis kyetä ennustamaan nykyisen tehtävän 
suorittamisen jälkeen. Tämän vuoksi artikkelin polunetsintäalgoritmt ajetaan 
uudestaan sille asetetun aikaikkunan $\omega$ mukaan. Tällöin algoritmeille 
riittää, että löydetyt polut ovat törmäysvapaita ainoastaan seuraavan 
aikaikkunan alkuun asti, eli korkeintaan $\omega$ ajan. Tätä kutsutaan 
artikkelissa rullaavan horisontin törmäystenratkaisuksi (Rolling Horizon 
Collision Resolution, RHCR) \par
	Artikkelin vertaamat polunetsintäalgoritmit ovat ikkunoitu 
hierarkinen yhteistoimintakykyinen A* (Windowed Hierarchical Cooperative A*, 
WHCA*), tehostettu konfliktipohjainen haku (Enhanced Conflict Based Search, 
ECBS) ja prioriteettipohjainen haku (Priority Based Search, PBS). WCHA* on 
hierarkinen (luku \ref{hpa}) A*-algoritmi, joka on muokattu toimimaan 
moniagentillisten ongelmien kanssa ikkunoidusti. ECBS on tehostettu versio 
CBS-algoritmista, joka etsii polkua iteroituvasti niin, että aina kun 
törmäyskohta löydetään niin polunetsintä aloitetaan alusta niin, että 
törmäysmahdollisuus huomioidaan rajoitteena. PBS on yhdistelmä 
moniagentilliseen ympäristöön mukautettua A*-algortimia ja CBS-algoritmia. 
Artikkelin mittauksissa PBS oli marginaalisesti tehokkaampi kuin WHCA* ja 
ECBS oli niin hidas, että se ei pystynyt tutkimukseen asennettujen 
aikarajoitteiden mukaan ratkaisemaan kuin yhden kolmesta tutkitusta 
varastosta.

\chapter{Eräiden algoritmien tehokkuuden tarkastelu esimerkkiongelmassa}\label{benchmarking}

Tässä tutkielmassa on verrattu kahta polunetsintäalgoritmia, Dijkstran 
algoritmia ja BFS:ää, ajamalla niitä koneellisesti monta kertaa satunnaisesti 
generoiduissa graafeissa ja tutkimalla ajautumisaikoja. Graafit on toteutettu
Boost Graph Library C++ kirjaston avulla jonka kehitti \textcite{bgl} . 
Testiohjelma ajaa algoritmit kahdessa Boost Graph Libraryn eri 
graafitoteutuksessa. Tämän tarkoitus on demonstroida graafien toteutuksen 
vaikutusta algoritmeihin. Testiohjelma itsessään on kehitetty tätä 
projektia varten ja julkaistu kokonaisuudessaan avoimen lähdekoodin jakeluun.
\cite{gt2} \par
	Testiohjelma ajaa jokaisella ajokerralla 2000 testiä, jossa jokaista 
varten generoidaan graafi $n$ solmulla ja $1,25n$ kaarella. Käytetyt 
graafikoot olivat $n=64$, $n=128$ ja $n=512$. Toteutuksen testisilmukassa 
oli kutsu, sekä BFS:lle, että Dijkstran algoritmille, sekä listatyyppisessä-, 
että matriisityyppisessä graafissa, mutta yksittäisissä testeissä 
kommentoitiin kaikki muu kuin testattava. Täten kun testattiin esimerkiksi 
Dijkstran algoritmia matriisityyppisessä graafissa, niin kaikki 
listatyyppisiin graafeihin ja BFS:ään liittyvät toimenpiteet oli kommentoitu 
pois, jolloin ne eivät vaikuta testien tuloksiin.\cite{gt2} \par

\begin{table}
\centering{}\caption{Yhteenveto testeissä mitatuista ajoista\label{tab:my-table1}}
\begin{tabular}{l|c|c|}
Graafien tyyppi ja koko (solmuja) & BFS & Dijkstran algoritmi \tabularnewline
\hline 
				Lista 64 	& 
\begin{tabular}{@{}c@{}}	Keskiarvo: 2,805 \\ Keskihajonta: 0,3972	\end{tabular} & 
\begin{tabular}{@{}c@{}}	Keskiarvo: 2,975 \\ Keskihajonta: 0,1565	\end{tabular} \tabularnewline
\hline
				Lista 128 	& 
\begin{tabular}{@{}c@{}}	Keskiarvo: 5,345 \\ Keskihajonta: 0,4766	\end{tabular} & 
\begin{tabular}{@{}c@{}}	Keskiarvo: 6,63  \\ Keskihajonta: 0,4840	\end{tabular} \tabularnewline
\hline
				Lista 512 	& 
\begin{tabular}{@{}c@{}}	Keskiarvo: 25,67 \\ Keskihajonta: 0,4714	\end{tabular} & 
\begin{tabular}{@{}c@{}}	Keskiarvo: 50,4  \\ Keskihajonta: 0,5931	\end{tabular} \tabularnewline
\hline
				Matriisi 64 	& 
\begin{tabular}{@{}c@{}}	Keskiarvo: 3,185 \\ Keskihajonta: 0,3893	\end{tabular} & 
\begin{tabular}{@{}c@{}}	Keskiarvo: 2,945 \\ Keskihajonta: 0,2286	\end{tabular} \tabularnewline
\hline 
				Matriisi 128 	& 
\begin{tabular}{@{}c@{}}	Keskiarvo: 9,49  \\ Keskihajonta: 0,5012	\end{tabular} & 
\begin{tabular}{@{}c@{}}	Keskiarvo: 8,77  \\ Keskihajonta: 0,4219	\end{tabular} \tabularnewline
\hline 
				Matriisi 512 	& 
\begin{tabular}{@{}c@{}}	Keskiarvo: 124,73 \\ Keskihajonta: 0,6156	\end{tabular} & 
\begin{tabular}{@{}c@{}}	Keskiarvo: 114,315 \\ Keskihajonta: 1,214	\end{tabular} \tabularnewline
\end{tabular}
\end{table}

Väliaikaista tekstiä, jotta näen miltä tulostaulukko näyttää takstin ympäröimänä:
Dipi diipa diipa doudou. Diipaa didi dou. Tyryry ryryryryry ryryry ryryry ry ryryryry turururu tururuu tu.
Dipi diipa diipa doudou. Diipaa didi dou. Tyryry ryryryryry ryryry ryryry ry ryryryry turururu tururuu tu.
Dipi diipa diipa doudou. Diipaa didi dou. Tyryry ryryryryry ryryry ryryry ry ryryryry turururu tururuu tu.
Dipi diipa diipa doudou. Diipaa didi dou. Tyryry ryryryryry ryryry ryryry ry ryryryry turururu tururuu tu.
Dipi diipa diipa doudou. Diipaa didi dou. Tyryry ryryryryry ryryry ryryry ry ryryryry turururu tururuu tu.
\par

Dipi diipa diipa doudou. Diipaa didi dou. Tyryry ryryryryry ryryry ryryry ry ryryryry turururu tururuu tu.
Dipi diipa diipa doudou. Diipaa didi dou. Tyryry ryryryryry ryryry ryryry ry ryryryry turururu tururuu tu.
Dipi diipa diipa doudou. Diipaa didi dou. Tyryry ryryryryry ryryry ryryry ry ryryryry turururu tururuu tu.
Dipi diipa diipa doudou. Diipaa didi dou. Tyryry ryryryryry ryryry ryryry ry ryryryry turururu tururuu tu.
Dipi diipa diipa doudou. Diipaa didi dou. Tyryry ryryryryry ryryry ryryry ry ryryryry turururu tururuu tu.


\chapter{Yhteenveto}\label{yhteenveto}

Tämän tutkielman tutkimuskysymys \ref{tKysymys1} oli selvittää minkälaisia 
polunetsintäalgoritmeja on kehitetty. Polunetsintäalgoritmit voidaan muun 
muassa jakaa yksiagentillisiin ja moniagentillisiin polunetsintäalgoritmeihin 
sen mukaan monellekko etsintäalueella liikkuvalle agentille ollaan etsimässä 
reittiä. Suurin osa tämän tutkielman luvussa \ref{joitainP} käsitellyistä 
algoritmeista on yksiagentillisia. Moniagentillisista algoritmeista puhutaan 
lähinnä luvussa \ref{robotiikka} robotiikan kontekstissa. \par
	Polunetsintäalgoritmit voidaan jakaa myös epäinformoituihin, 
informoituihin ja metaheuristisiin algoritmeihin. Näistä epäinformoidut 
algoritmit eivät ole tietoisia etsimänsä alueen yksityiskohdista, kun taas 
informoidut algoritmit voivat käyttää etsintäalueesta laskemaansa dataa 
nopeuttamaan algoritmia. Metaheuristiset algoritmit taas käyttävät muita 
keinoja kuin solmukohtien tutkimista tai heuristisia funktioita polkujen 
etsimiseen.\cite{applSciLawande} Nämä on rajattu tästä tutkielmasta pois. \par
	Näiden lisäksi on olemassa myös hierarkisia ja dynaamisia 
polunetsintäalgoritmeja. Hierarkiset polunetsintäalgoritmit luovat 
etsintäalueesta yksinkertaistetun abstraktion, jossa on helpompi ratkaista 
ongelma. Varsinaisen ongelman ratkaisu johdetaan sitten abstraktion 
ratkaisusta.\cite{rda} Hierarkisesta polunetsinnästä puhutaan tarkemmin 
luvussa \ref{hpa}. Dynaamiset algoritmit taas kykenevät reagoimaan 
etsintäalueessa tapahtuviin muutoksiin reaaliajassa. Näistä puhutaan 
pikaisesti luvussa \ref{robotiikka}. \par
	Tutkimuskysymys \ref{tKysymys2} oli selvittää miten 
polunetsintäalgoritmeja voidaan käyttää käytännön sovelluksiin. Tähän 
kysymykseen vastataan kappaleessa \ref{algoritmienSovelluskohteita}, jossa 
syvennytään tarkemmin kolmeen sovelluskohteeseen: videopeleihin, 
karttaohjelmistoihin ja robotiikkaan. \par
	Tutkimuskysymys \ref{tKysymys3} oli, että miten 
polunetsintäalgoritmien tehokkuutta voidaan mitata. Tätä varten toteutettiin 
pienimuotoinen tutkimus, joka sisälsi C++-toteutukset kahdesta algoritmista, 
joiden ajoaikoja vertailtiin. Tämän tutkimuksen tulokset ja ohjelmakoodi 
on julkaistu verkossa.\cite{gt2} Tutkimuksissa todettiin, että graafin 
tekninen toteutus vaikutti ajoaikoihin enemmän kuin käytetty algoritmi. 
Hiukan yllättäen Dijkstran algoritmi ei ollut tutkimuksessa kaikkialla 
nopeampi kuin BFS. Tämä ei vastaa muiden vastaavien kokeiden tuloksia.
\cite{mazeGameTrilogi} Muissa vastaavissa mittauksissa on mitattu ajoajan 
lisäksi löydettyjen polkujen pituutta ja laskutoimituksien lukumäärää.
\cite{mazeGameTrilogi}\cite{pPacman} \par
	Minun tekemää testiohjelmaa voisikin parantaa lisäämällä siihen 
ajoajan lisäksi muita mittareita. Lisäksi voisin jatkotutkimuksena selvittää 
miksi Dijkstran algoritmi oli testiohjelmassa tietyissä tietorakenteissa 
hitaampi kuin BFS. Mikäli se johtuu tietorakenneoperaatioiden hitauseroista 
niinkuin luulen sen johtuvan, niin saattaisi olla järkevää verrata näitä 
algoritmeja uudestaan kullekin tietorakenteelle optimoidulla algoritmilla 
geneerisen algoritmin sijasta. \par
	Tässä tutkielmassa tutustuttiin polunetsintäalgoritmeihin, sekä 
niiden käyttökohteisiin ja toteutettiin pienimuotoinen tutkimus niiden 
vertailemiseksi. Tutkimuskysymyksiin on vastattu ja mahdollisia tulevan 
tutkimisen aiheita on löydetty.


% The thesis main content ends here.

\printbibliography

\begin{comment}
Important! Create the appendix chapters with command \textbackslash appchapter\{some
name\} instead of \textbackslash chapter\{some name\} for the automagic
page counting to work!
\end{comment}


\appchapter{Liitedokumentti placeholder}

\textit{*Placeholder liitedokumenteille*}

\end{document}
