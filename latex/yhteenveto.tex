\chapter{Yhteenveto}\label{yhteenveto}

Tämän tutkielman tutkimuskysymys \ref{tKysymys1} oli selvittää minkälaisia 
polunetsintäalgoritmeja on kehitetty. Polunetsintäalgoritmit voidaan muun 
muassa jakaa yksiagentillisiin ja moniagentillisiin polunetsintäalgoritmeihin 
sen mukaan monellekko etsintäalueella liikkuvalle agentille ollaan etsimässä 
reittiä. Suurin osa tämän tutkielman luvussa \ref{joitainP} käsitellyistä 
algoritmeista on yksiagentillisia. Moniagentillisista algoritmeista puhutaan 
lähinnä luvussa \ref{robotiikka} robotiikan kontekstissa. \par
	Polunetsintäalgoritmit voidaan jakaa myös epäinformoituihin, 
informoituihin ja metaheuristisiin algoritmeihin. Näistä epäinformoidut 
algoritmit eivät ole tietoisia etsimänsä alueen yksityiskohdista, kun taas 
informoidut algoritmit voivat käyttää etsintäalueesta laskemaansa dataa 
nopeuttamaan algoritmia. Metaheuristiset algoritmit taas käyttävät muita 
keinoja kuin solmukohtien tutkimista tai heuristisia funktioita polkujen 
etsimiseen.\cite{applSciLawande} Nämä on rajattu tästä tutkielmasta pois. \par
	Näiden lisäksi on olemassa myös hierarkisia ja dynaamisia 
polunetsintäalgoritmeja. Hierarkiset polunetsintäalgoritmit luovat 
etsintäalueesta yksinkertaistetun abstraktion, jossa on helpompi ratkaista 
ongelma. Varsinaisen ongelman ratkaisu johdetaan sitten abstraktion 
ratkaisusta.\cite{rda} Hierarkisesta polunetsinnästä puhutaan tarkemmin 
luvussa \ref{hpa}. Dynaamiset algoritmit taas kykenevät reagoimaan 
etsintäalueessa tapahtuviin muutoksiin reaaliajassa. Näistä puhutaan 
pikaisesti luvussa \ref{robotiikka}. \par
	Tutkimuskysymys \ref{tKysymys2} oli selvittää miten 
polunetsintäalgoritmeja voidaan käyttää käytännön sovelluksiin. Tähän 
kysymykseen vastataan kappaleessa \ref{algoritmienSovelluskohteita}, jossa 
syvennytään tarkemmin kolmeen sovelluskohteeseen: videopeleihin, 
karttaohjelmistoihin ja robotiikkaan. \par
	Tutkimuskysymys \ref{tKysymys3} oli, että miten 
polunetsintäalgoritmien tehokkuutta voidaan mitata. Tätä varten toteutettiin 
pienimuotoinen tutkimus, joka sisälsi C++-toteutukset kahdesta algoritmista, 
joiden ajoaikoja vertailtiin. Tämän tutkimuksen tulokset ja ohjelmakoodi 
on julkaistu verkossa.\cite{gt2} Tutkimuksissa todettiin, että graafin 
tekninen toteutus vaikutti ajoaikoihin enemmän kuin käytetty algoritmi. 
Hiukan yllättäen Dijkstran algoritmi ei ollut tutkimuksessa kaikkialla 
nopeampi kuin BFS. Tämä ei vastaa muiden vastaavien kokeiden tuloksia.
\cite{mazeGameTrilogi} Muissa vastaavissa mittauksissa on mitattu ajoajan 
lisäksi löydettyjen polkujen pituutta ja laskutoimituksien lukumäärää.
\cite{mazeGameTrilogi}\cite{pPacman} \par
	Minun tekemää testiohjelmaa voisikin parantaa lisäämällä siihen 
ajoajan lisäksi muita mittareita. Lisäksi voisin jatkotutkimuksena selvittää 
miksi Dijkstran algoritmi oli testiohjelmassa tietyissä tietorakenteissa 
hitaampi kuin BFS. Mikäli se johtuu tietorakenneoperaatioiden hitauseroista 
niinkuin luulen sen johtuvan, niin saattaisi olla järkevää verrata näitä 
algoritmeja uudestaan kullekin tietorakenteelle optimoidulla algoritmilla 
geneerisen algoritmin sijasta. \par
	Tässä tutkielmassa tutustuttiin polunetsintäalgoritmeihin, sekä 
niiden käyttökohteisiin ja toteutettiin pienimuotoinen tutkimus niiden 
vertailemiseksi. Tutkimuskysymyksiin on vastattu ja mahdollisia tulevan 
tutkimisen aiheita on löydetty.
